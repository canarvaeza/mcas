\chapter{Introducci\'{o}n}
\label{chp:Introduccion}

\section{Justificación}
\label{sec:Justificacion}

\section{Problema}
\label{sec:Problema}

Perera [9] señala la importancia de definir estrategias que se concentren en el modelado, razonamiento y descubrimiento de las variables de contexto. Adicionalmente los métodos que permiten extrae datos de contexto a partir de anotaciones y comentarios en documentos multimedia (imágenes, audio y texto) son poco trabajados, normalmente las fuentes de información son los sensores que incluyen los dispositivos de interacción o las preferencias que el usuario provee al iniciar el uso de un sistema.
Dado lo anterior, es necesario el diseño de una estrategia que permita realizar el modelado de contexto a partir de la información contenida en las anotaciones semánticas de diferentes formatos [26] y razonar sobre el modelo para generar conocimiento que permita personalizar los contenidos presentados a los usuarios en diferentes contextos [12].

Dado lo anterior se plantea la pregunta de investigación \textit{\textbf{¿Cómo puede la información obtenida a partir del análisis semántico de las anotaciones en contenidos digitales, mejorar los procesos de un sistema sensible al contexto?}}.

\section{Objetivos}
\label{sec:Int_Objetivos}

\section{Metodología}
\label{sec:Int_Metodologia}

Para alcanzar los objetivos propuestos en esta tesis de maestría se han realizado las siguientes actividades con respecto a los objetivos planteados en la sección \ref{sec:Int_Objetivos}
\begin{itemize}
    \item Actividades relacionadas con el obj 1
    \begin{itemize}
        \item Estudio de ....
    \end{itemize}
\end{itemize}