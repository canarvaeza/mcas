\chapter{Marco Teorico}
\label{chp:Marco_Teorico}

En este capítulo se presentan, de forma general, los fundamentos teóricos más importantes para el desarrollo de la propuesta. Entre los fundamentos también se
analizan las problemáticas más importantess en cada área y su pertinencia para este proyecto.

\section{Contenidos Multimedia}
\label{sec:MT_ContenidosMultimedia}

En la actualidad los seres humanos interactúan con una gran variedad de dispositivos y aplicaciones y la cantidad de información digital distribuida por medio de internet ha incrementado notablemente \cite{stamou2006multimedia}. Adicionalmente la masificación de los celulares inteligentes ha cambiado la forma en la que los seres humanos interactúan usando la tecnología, haciendo común el uso de combinaciones de datos multimedia \cite{bracamonte2017extracting}. Multimedia se define como la combinación de diferentes medios digitales, codificados en archivos, con el fin de explicar, representar o demostrar un concepto \cite{rowe2005acm}, dentro de esta categoría se pueden encontrar las imágenes, el texto, los audios, las gráficas, los vídeos, entre otros.
En el año 2004 el Grupo de Interés Especial en Multimedia de ACM definió como uno de los retos de investigación en multimedia hacer de la captura, almacenamiento, recuperación y \textbf{uso} de los contenidos un tema recurrente en los entornos computacionales \cite{rowe2005acm}. Precisamente la proliferación de los dispositivos móviles y del acceso a internet ha permitido que personas alrededor del mundo generen contenidos que varían en características como formato y contexto \cite{bracamonte2017extracting}. Sistemas como Facebook proporcionan a los usuarios herramientas para generar contenidos que transmiten información de forma masiva; los contenidos compartidos en esta plataforma pueden ser de tipo texto, imagen o video, y pueden ser creados o compartidos por el usuario desde otras plataformas.
Para ACM \cite{rowe2005acm} la indexación de contenido, la anotación de datos y la obtención de la semántica asociada; son procesos fundamentales para facilitar la búsqueda y uso de los datos multimedia. Generalmente, se crean dos tipos de información para los datos multimedia: el contenido y los metadatos \cite{bracamonte2017extracting}; a su vez, las anotaciones incluidas en los metadatos son de dos clases: las que describen los atributos del contenido (anotaciones de bajo nivel) y las que relacionan la semántica de los conceptos representados por esos contenidos (anotaciones de alto nivel), la variedad en los datos multimedia hace más compleja la búsqueda y comprensión de los contenidos por parte de las máquinas o agentes de software.

#Y aunque en la actualidad las tecnologías en la nube y la capacidad de procesamiento han dado solución a algunos problemas, varios autores han identificado la importancia de agregar datos de contexto a los servicios que manejan los usuarios día a día \cite{dey2001understanding, alegre2016engineering}.
# En el trabajo propuesto por Narváez et al. \cite{narvaez2016modelo} se proponen cuatro componentes para representar un contenido  multimedia: (i) Objeto Binario, es el archivo original. (ii) Objeto Archivo, contiene los metadatos del formato. (iii) Objeto Multimedia, contiene los descriptores asociados al contenido. Y (iv) Objeto Semántico, contenedor estructurado de los conceptos semánticos vinculados a una entidad. La distribución anterior permite la recuperación de los contenidos desde diferentes perspectivas y por lo tanto es posible analizar, recuperar y sugerir diferentes elementos al mismo tiempo.

\subsection{Metodos de Análisis de Contenidos Multimedia}

Dependiendo del formato a analizar y de la información que se quiere obtener (normalmente anotaciones de bajo nivel) se cuenta con diferentes técnicas, a continuación se presenta un listado de técnicas para cada uno de los formatos.
REVISAR QUE MÉTODOS PERMITEN NO SOLO ENCONTRAR INFORMACIÓN DE BAJO SINO TAMBIÉN DE ALTO NIVEL
\begin{itemize}
    \item Imagen:
    \item Vídeo:
    \item Audio:
    \item Texto:
\end{itemize}

Todos estos metodos pueden ser combinados con algoritmos de Inteligencia Artificial como ----------------- para obtener anotaciones de alto nivel, estas anotaciones tienen como principal objetivo ofrecer .....

#\subsection{Anotación Semántica de Contenidos Multimedia}

Los siguientes métodos son pupulares para la anotación semántica de contenidos multimedia


\section{Sensibilidad al Contexto}
\label{sec:MT_SensibilidadContexto}

El contexto ha sido definido por diferentes autores, pero la definición más importante y ampliamente aceptada ha sido la de Dey en 2001 quien dice \textbf{"El contexto es cualquier información que puede ser usada para caracterizar la situación de una Entidad. Una entidad es una persona, lugar, u objeto considerada relevamte para la interacción existente entre el usuario y una aplicación, esto incluye a la aplicación y al usuario."}. Entonces un sistema sensible al contexto es \textbf{aquel que usa el contexto para proveer servicios e información relevante al usuario, en donde la relevancia depende de la actividad que desarrolla el usuario"}  \cite{Dey2001}.
Para ello existe un proceso que se divide en las siguientes etapas \cite{Perera2014}:

\begin{itemize}
    \item Adquisicón del Contexto: Se obtienen los datos de contexto desde diferentes fuentes de información, normalmente datos de sensor.
    \item Modelado del Contexto: Selección de los datos de contexto relevantes para el sistema.
    \item Razonamiento del Contexto: Permite generarción conocimiento a partir de la información de contexto disponible, obteniendo contexto de alto nivel.
    \item Distribución del Contexto: Servicios ricos en contexto para las entidades del sistema.
\end{itemize}

Las fuentes de datos de los Sistemas Sensibles al Contexto se dividen en tres típos físicos, virtuales y lógicos; siendo los físicos los que se deriban directamente de los sensores, los virtuales los que se obtienen de servicios externos a la aplicación y los lógicos combinan las dos típos de datos. Aunque el contexto generado por medio de fuentes visuales, de audio y de movimiento hacen parte de los contenidos multimedia; los desarrollos sobre estos sistemas se han concentrado en el uso de sensores de localización, aceleración y orientación; y servicios de información climatológica y descripción de los lugares de interés. El uso de contenidos multimedia no se exploran en muchos trabajos del estado del arte, este tema se trata a profundidad en la sección \ref{LA SECCIÓN}

El análisis de los datos generados por los sensores en el sistema permite producir el \textbf{Contexto de Bajo Nivel} el cual representa actividades simples y triviales. La adición de una capa de razonamiento dota al sistema con la capacidad de producir \textbf{Contexto de Alto Nivel} . De forma general se puede interpretar que los datos de localización obtenidos por un sensor GPS permiten saber a la aplicación que la persona se encuentra en un gimnasio, un pulsometro identifica la actividad comer, y finalmente un proceso de razonamiento permite comprender que el usuario está desarrollando su rutina de ejercicio.
Dentro de los sistemas sensibles al contexto la actividad de generar información contextual de alto nivel es de gran importancia, los procesos de Modelado y Razonamiento tienen técnicas variadas que pueden ser seleccionadas de acuerdo al sistema y los objetivos que deben cumplir. En cuanto al modelado, las ontologías y las basadas en objetos son las más usadas; mientras que en el razonamiento son las reglas y los arboles de decisión las más populares. En la sección \ref{sec:MT_Ontologias} se describen las ontologías y en el artículo \cite{A Survey of Context Modelling and Reasoning Techniques} se presentan el resto de los métodos de forma más específica.

#Problemas de la SC \cite{Alegre2016}


\section{Ontologías}
\label{sec:MT_Ontologias}

Una ontología permite representar formalmente el conocimiento de un dominio por lo tanto es una técnica de modelado de información muy versatil que organiza la información y las entidades que la representa.
Algunas de las ventajas que tienen las ontologías son (i) Compartir una estructura definida entre sistemas o personas. (ii) Reutilizar el conocimiento conseguido en un dominio específico. (iii) Hacer explícitas las suposiciones. (iv) Separar el conocimiento de un dominio de las estratégias operativas. En el área de la sensibilidad al contexto el apartado (ii) es importante pues permite hacer uso de información desde diferentes fuentes.

Las ontologías definen la información en tripletas compuestas por un sujeto (Persona) una propiedad (tiene) y un objeto (la pelota), adicionalmente presenta relaciones como la pertenencia de un concepto a un grupo específico o la posibilidad de representar entidades que tienen el mísmo significado semántico. La W3C con su estandar OWL \cite{CITA W3C OWL} permite representar ontologías para sistemas computacionales, está basado en RDF y es el estandar más usado en el área.
En la sección \ref{subsubsec:Prop_Mod_mContext} se presenta la ontología de contexto multimedia.

\section{Web Semantica}
\label{sec:MT_WSemantica}

La web semantica es ...

\subsection{SPARQL}
SPARQL es un lenguaje de consulta para grafos RDF.

\section{Linked Data}
\label{sec:MT_LinkedData}
To be defined pues no creo que la use.