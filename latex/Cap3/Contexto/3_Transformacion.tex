\subsection{Transformación de datos}
\label{subsec:Prop_TransDatos}

Como se indica en la sección \ref{sec:MT_LinkedData} el uso de datos enlazados permite ....., adicionalmente en esta propuesta facilita el uso de fuentes de información que no se relacionan directamente con el dominio de la aplicación pero que pueden ser de utilidad para definir el contexto del usuario. El propósito principal de esta capa es transformar los datos binarios ( metadatos y etiquetas) a un formato estándar, como RDF, que facilite la interoperabilidad con otros sistemas que hacen uso de bases de conocimiento y la realización de consultas sobre estas. Al finalizar el proceso los datos estarán representados en RDF por lo cual será posible aplicar estrategias de razonamiento sobre ontologías que clasifiquen e identifiquen rápidamente el conjunto de reglas que se pueden operar.

\todo{Esto necesita una cita!}
El modelo principal de información para llevar a cabo esta transformación es la ontología mContext presentada en la sección \ref{subsubsec:Prop_Mod_mContext} y los datos a transformar son los especificados en la sección \ref{subsubsec:Prop_Adqui_Datos}. Los métodos para realizar esta actividad se basan en el estándar R2RML\footnote{Recomendación W3C: http://www.w3.org/TR/r2rml/} el cual permite transformar contenidos de bases de datos al formato RDF, esto se hace encontrando la relación entre los campos de la base de datos, y las clases y propiedades que tiene una ontología. La necesidad de transformar datos que tienen formatos y orígenes diferentes a los de las bases de datos, como los resultados del llamado a un servició onlíne, han llevado al desarrollo de herramientas como RML \footnote{http://rml.io} el cual permite usar formatos JSON, CSV, ... como origen y luego transformar a RDF teniendo en cuenta los recursos ontológicos que el usuario haya desarrollado o que estén disponibles en la red.





% El uso de datos enlazados en los sistemas sensibles al contexto no ha sido explorado por muchos autores, entre los trabajos realizados se encuentra *****, estos trabajos concluyen *****. La aplicación del paradigma de datos enlazados es una parte importante del desarrollo de este trabajo de investigación pues ----. El proceso general de transformación de datos se puede encontrar a continuación.

\subsubsection{Proceso de Linked Data}
\label{subsubsec:Prop_TransData_LD}
ES NECESARIO DEFINIR ESTO ENTRE ESTA Y LA SIGUIENTE SEMANA